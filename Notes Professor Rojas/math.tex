%%Credit: Chris Thomson github: https://github.com/christhomson/lecture-notes

\documentclass[]{article}
\usepackage[margin = 1.5in]{geometry}
\setlength{\parindent}{0in}
\usepackage{amsmath}
\usepackage{amsthm}
\usepackage{amsfonts}
\usepackage{amssymb}
\usepackage{mathtools}
\usepackage{wasysym}
\usepackage{soul}
\usepackage[T1]{fontenc}
\usepackage{ae,aecompl}
\usepackage{tikz}
\usepackage[lined]{algorithm2e}
\usepackage{enumerate}
\usepackage[super,negative]{nth} %nth
\usepackage[pdftex,
  pdfauthor={Carlos Osco Huaricapcha},
  pdftitle={Solving System of Polynomial Equations},
  pdfsubject={Lecture notes from Algebraic Topology at MSRI},
  pdfkeywords={course notes, notes, Mathematics Science Research Institute, System of Polynomials Equations, combinatorics, mathematics},
  pdfproducer={LaTeX},
  pdfcreator={pdflatex}]{hyperref}

\usepackage{cleveref}

\newtheorem*{theorem}{Theorem}
\newtheorem*{corollary}{Corollary}
\newtheorem*{proposition}{Proposition}

\theoremstyle{definition}
\newtheorem{problem}{Problem}[section]
\newtheorem*{defn}{Definition}
\newtheorem{ex}{Example}[section]

\newcommand{\union}{\cup}
\newcommand{\intersection}{\cap}

\setlength{\marginparwidth}{1.5in}
\newcommand{\lecture}[1]{\marginpar{{\footnotesize $\leftarrow$ \underline{#1}}}}

\DeclarePairedDelimiter{\set}{\lbrace}{\rbrace}

\crefname{ex}{Example}{Examples}
\crefname{proposition}{Proposition}{Propositions}

\definecolor{darkish-blue}{RGB}{25,103,185}

\hypersetup{
    colorlinks,
    citecolor=darkish-blue,
    filecolor=darkish-blue,
    linkcolor=darkish-blue,
    urlcolor=darkish-blue
}


\tikzset{
	dot/.style 2 args={circle,inner sep=2pt,fill,label=#1:{#2},name=#2},
	rdot/.style 2 args={circle,inner sep=2pt,fill=red,label=#1:{#2},name=#2},
	gdot/.style 2 args={circle,inner sep=2pt,fill=green,label=#1:{#2},name=#2},
	bdot/.style 2 args={circle,inner sep=2pt,fill=blue,label=#1:{#2},name=#2}
}

\makeatletter
\def\blfootnote{\gdef\@thefnmark{}\@footnotetext}
\makeatother

\begin{document}
	\let\ref\Cref
	\title{\bf{Solving System of Polynomial Equations}}
	\date{Summer 2017, Mathematics Science Research Institute \\ \center Notes written from Maurice Rojas' lectures.}
	\author{Carlos Osco Huaricapcha}

	\blfootnote{I'd love to hear your feedback. Feel free to email me at \href{mailto:coscohua@mail.sfsu.edu}{coscohua@mail.sfsu.edu}.}

%	\blfootnote{See \href{http://cthomson.ca/notes}{cthomson.ca/notes} for updates.
%		\ifdefined\sha % Also, \commitDateTime should be defined.
%		%Last modified: \commitDateTime{} ({\href{https://github.com/christhomson/lecture-notes/commit/\sha}{\sha}}).
%		\fi
%	}

	\maketitle
	\newpage
	\tableofcontents
	\newpage

	\section{Introduction, Algebraic Geometry} \lecture{June 26, 2017}
%		\subsection{Course Structure}
%			The grading scheme is \st{50\%} 55\% final, 30\% midterm, \st{10\% quizzes} 5\% participation marks (clicker questions), and 10\% assignments. There are \st{ten quizzes and} ten assignments. \st{The quizzes will not be announced in advance, and each quiz consists of a single clicker question}. Assignments are typically due on Friday mornings at 10 am. in the dropboxes outside of MC 4066. The midterm exam is scheduled for March 7, 2013, from 4:30 pm - 6:20 pm. There is no textbook for the course, but there are course notes available at Media.doc in MC, and they're \emph{highly} recommended.
%			\\ \\
%			MATH 239 is split into two parts: counting (weeks 1-5) and graph theory (weeks 6-12).
%			\\ \\
%			See the course syllabus for more information -- it's available on \href{https://learn.uwaterloo.ca/}{Waterloo LEARN}.

		\subsection{What is it?}
		Ideally, Algebraic Geometry occurs over rings other than $\mathbb{C}$. Let's see some examples of polynomial solving.
			\begin{ex}
			Let $N=pq$ where $N$ is a positive integer and $p,q >1$. (Factoring.)
			Usually, we are given $N$, so we can find $p$ and $q$.\\
			The security of the R.S.A. Crypto-system is based on this \emph{hardness}.
			\end{ex}
			\begin{ex}
			Equilibria in Chemical Analysis systems (reaction networks).
			\begin{align*}
			\text{linear ordinary differential equation} \to &&\dot{x} = 3x+7\\
			\text{non-linear ordinary differential equation} \to &&\dot{x} = \text{polynomial}(x) (\text{deg}\geq 2)
			\end{align*}
			\end{ex}
			Differential equations govern chemical reaction rates. So, when the reaction settles down the concentrations reactions an \emph{equilibrium}, and the concentrations wind up being \emph{real} solutions to a system of polynomial equations.
			It is fair to say that \emph{real solving} in fact makes up to more than 30\% of electrical engineering.
			\begin{ex}
			$\mathbb{F}_1$: Finite field with $q$ elements in coding theory and cryptography. You often do arithmetic over $\mathbb{F}_q$, and polynomial system solving occurs very often. ($q$=prime power).
			\end{ex}
			\begin{ex}
			$\mathbb{Q}_p$($p$-acid reactions): This field is related to the solution of Weil-conjectures around the mid-20th century. Multiple fields medals were ensued.\\
			\end{ex}
			Fun fact: 40\% of the field medals are awarded to Algebraic Geometry.
			\\
			Let's start with easy equations (1 equation, 1 variable, 2 terms, $\mathbb{C}$).
			\begin{ex}
			Given $c_1,c_2 \in \mathbb{C}; a_1,a_2 \in \mathbb{Z}$. How do we solve
			\[ c_1x_1^{a_1}+c_2x_1^{a_2}=0\]?
			Notice that it is easy to reduce to the case
			\[ x^d=c \quad \text{where} \quad d\in\mathbb{Z}; c\in \mathbb{C}.\]
			In fact if you throw-out $x_1=0$, then we may assume $c\neq0$.
			\end{ex}
			%Eulers
			\subsection{Euler's Formula}
			It is easy to take d\textsuperscript{th} roots.
			\[ \exp^{\sqrt{-1}\tau}= \cos\tau+\sqrt{-1}\sin\tau.\]
			\textbf{Exercise 1.}
			What is a simple formula for the roots of $x_1^{d}=c$ using Euler's formula?
			\underline{Important observation:} Polynomials and polytopes are long lost siblings.
			%ArcNewt
			\subsection{ArchNewt and Newt}
			\begin{defn}
			$ArchNewt(c_1x_1^{a_1}+c_2x_2^{a_2}$):=
			\[Pic goes here\]
			Consider the slope of this line segment(assuming $a_1\neq a_2$).
			\begin{align*}
			\frac{\log|c_2|-(-\log|c_1|)}{a_2-a_1} = \log\left| \text{all the roots of } c_1x_1^{a_1}+c_2x_2^{a_2}\right|.
			\end{align*}
			\end{defn}
			%Exercise 2
			\textbf{Exercise 2.} Prove it.\\
			
			Here is another connection: If $f(x)=c_0+c_1x_1+\ldots+c_dx_1^d.$ where $d\geq1$ is the degree then $f$ has exactly $d$ roots counting multiplicity. (A version of the Fundamental Theorem of Algebra ``F.T.A'').
			This fact goes back to 1609 ``P. Roth,'' and 1629 ``Argand.''
			%%Newt
			\begin{defn}
			$Newt(c_o+c_1x_1+\ldots+c_dx_1^{d}):= [0,d].$
			\\
			Ray pic goes here.\\
			
			\underline{Observe:} $d=Length(Newt(\ldots))$.
			\end{defn}
			\subsection{Backup}
			\subsubsection{Convex}
			\begin{defn}
			Given any $S \subseteq \mathbb{R}^n$. We say $S$ is convex $\iff$ for all $x,y \in S$ the line segment connecting $x$ and $y$ also lies in $S$.
			\[ \{ \lambda x + (1-\lambda)y \mid \lambda \in [0,1]\}.\]
			\end{defn}
			\subsubsection{Convex Hull}
			\begin{defn}
			Given points $a1,\ldots,a_\tau \in \mathbb{R}^n$ their convex hull is
			\begin{align*}
			Conv(\{a1,\ldots,a_\tau\}):= \text{smallest convex set containing} a1,\ldots,a_\tau.
			\end{align*} 
			\end{defn}
			\begin{ex}
			$Conv(\{\ddots\}$) = Line.\\
			In $\mathbb{R}^3$ it is the wrapping ribbons around one point to another (?).
			\end{ex}
			\subsubsection{Polytope}
			\begin{defn}
			A polytope (in $\mathbb{R}^n$) is just the convex hull of any finite point set in $\mathbb{R}^n$.
			\end{defn}
			\subsubsection{Correct definitions for Newt and Archnewt}
			Let's think about column vectors. If $c_1x^{a_1}+\dots+c_Tx^{a_\tau} \quad (c_1,\dots,c_\tau \in \mathbb{C}^{*})$ where $X=(x_1,\dots,x_n)$ and $X^{a_{J}}=x_1^{a_1,J},\dots, X_n^{a_n,J}$. Then,
			\begin{align*}
			Newt(f) &: = \text{Conv}(\{a_1,\ldots,a_\tau\}) \\
			ArchNewt(f) &: = \text{Conv}(\{(a_J,-\log|c_J|) \mid \text{\scriptsize{J}} \in \{1,\ldots,\tau\}\}).
			\end{align*}
			\textbf{Exercise 3.} Find each of following: $Newt(0), Newt(1-x_1), Newt(1-x_1+x_2+x_1x_2)$, and $ArchNewt(1+1000x_1+x_1^3)$.
						\vspace{.25in}
			\\
			\underline{Note:} 
			\begin{align*}
			\mathcal{A} &:= [a_1,\ldots, a_\tau]\\
			&= \left[\begin{array}{ccc}a_{1,1} & \ldots & a_{1,\tau} \\\vdots & \ddots & \vdots \\a_{n,1} & \dots & a_{n,\tau}\end{array}\right] \in \mathbb{Z}^{n\times\tau}
			\end{align*}
			is called the support of $f$.
			\subsection{Univariate Trinomials}
			Let $f(x_1)=1-1000x^18+x_1{51}$. What can we say about the $(\mathbb{C})$ roots?
			
			\begin{center}``Physicists Approach\ldots'': $|x_1|
			\begin{cases}
			\text{small}\\
			\text{big}
			\end{cases}
			$\end{center}
			
			When $|x_1|$ is \emph{small:} $|x_1|^{18} \gg|x_1|^{51}$. Perhaps, the roots of $1-1000x_1^{18}$ are close to the roots of $1-1000x_1^{18}+x_1^{51} \Rightarrow$ maybe (?) there are \underline{18} roots of $f$ with norm $\sqrt[18]{\frac{1}{1000}}$ (small radius).
			
			Also, when $|x_1|$ is \emph{big:} $|x_1|^{51}\gg|x_1|^{18} \Rightarrow$ maybe(?) the nonzero roots of $1-1000x_1^{18}+x_1^{51}$ are near the nonzero roots of $f$. Perhaps, $f$ has 33 roots of norm near $\sqrt[33]{1000}$ (big radius).
			\\
			
			\underline{Truth:} 18 roots near inner orbit and 33 in the bigger orbit.\\
			
			\vspace{.25in}
				pic goes here of the orbits			
			\vspace{.25in}\\
			
			\textbf{Exercise 4.} How close are the norms of the roots of $f$ to the 2 approximations?\\
			
			\underline{Observe:} Archnewt$(f)$ is the following graph closed by the points $(0,0),(51,0),$ and $(18,-\log1000)$.\\
			
			\vspace{.25in}
			Graph of the triangle goes here
			\vspace{.25in}\\
			slope ``$s$'' of the lowets segements are close to the $\log|\text{roots of $f$|}$.
			
			\section{Archimedean Tropical Variety}
			\subsection{ArchTrop and Amoeba}
			The connection between polynomials and polytopes is closely related to Tropical Geometry. Historically, the roots of the connection go back to Newton around 1676, I. Newton wrote a letter describing how to extract power series solutions $y=y(x)$ to polynomial equations like $f(x,y)=0$.\\
			Also, J. Hadamard around 1896, derived a version of $ArcNewt$ for the power series, and observed a connection between slopes of edges and location of roots.
			\begin{defn} For one variable:
			\begin{align*}
			ArchTrop(f)= \text{slopes of lower edges of $ArchNewt(f)$}.
			\end{align*}
			\end{defn}
						\begin{defn}
			\begin{align*}
			Amoeba(f)= \{\log|J| \mid J \in \mathbb{C}, f(J)=0, J\neq 0\}.
			\end{align*}
			\end{defn}
			\begin{theorem}[Avenda\~{n}o, Kogan, Nisse, Rojas.(2013)] If $f(x_1)=c_1x^{a_1}+\dots+c_Tx^{a_\tau}$ then, for any $u \in Amoeba(f)$ there is a $v\in ArchTrop(f)$ with $|u-v| \leq \log 3$.
			\end{theorem}
			 I.e. These sets approximate each other.
			\begin{ex} Given: $T=2$(binomials). Then, $Amoeba = $ 1 point and $ArchTrop=$ 1 point.
			\end{ex}
			\begin{ex} Given: $T=3$(trinomials). Then, $Amoeba = $ up to the degree many points points. $1-1000x^{18}+x^{51}$ has $\approx$ 26 points (?) . Also, $ArchTrop(f)=$ 2 points. (Include graph of numb line).
			\end{ex}
			\textbf{Exercise 5.} Figure out how many points.\\
			\subsection{Vertex, Edge, and Lower Edge}
			\subsubsection{Vertex}
			\begin{defn}
			Consider minimizing $\alpha x+\beta y$ for $(x,y)\in P.$ If this minimum is attained at a unique point $v$ then we call $v$ a \emph{vertex} of $P$ with inner normal $(\alpha,\beta)$.
			\end{defn}
			\subsubsection{Edge}
			\begin{defn}
			Consider minimizing $\alpha x+\beta y$ for $(x,y)\in P.$ If this minimum is attained at a proper subset of $P$(with $\geq$2 points) then this set is called an \emph{edge} with inner normal $(\alpha,\beta)$. (Include graph of norms).
			\end{defn}
			\subsubsection{Lower Edge}
				\begin{defn}
			If $P \in \mathbb{R}^2$ is a polygon then $E \subset P$ is a lower edge if and only if $E$ is an edge with inner normal $(\alpha,\beta)$ for some $(\beta >0)$. (Include graph of polygon).
			\end{defn}
			\section{ArchTrop and Amoeba in multivariable functions}
			Consider $f(x_1,x_2)=1+x_1+x_2$.
			\begin{defn}
			If $f$ is any polynomial in $x=(x_1,\ldots,x_n)$ then 
			\[Amoeba(f):= \{(\log|J|,\ldots,\log|J_n|) \mid f(J_1,\dots J_n)=0; J_1,\dots J_n \in \mathbb{C}^{*} \}.\]
			\end{defn}
			\begin{defn}
			%\begin{align*}
			\[ArchTrop(f):= \{v\in \mathbb{R}^{n} \mid \text{is an inner normal to a face of $ArchNewt(f)$ of positive dimension}\}.\]
			%\end{align*}
			\end{defn}
			\underline{Note:} We will clarify faces and dimension in complete generality next time. For now, let`s consider the $Amoeba$ .
			\[ Enter pic of Amoeba here\]
			Amoebae are useful because they give some intuition to begin higher dimensional zero sets.\\
			\underline{Note:} $f(x_1,x_2)$ where $x_1,x_2 \in \mathbb{C}$ has 2 manifold in $\mathbb{R}^{+}$.\\
			Let's analyze why is that picture true.
			\begin{center}
			If $ 1+x_1+x_2=0$ then $\begin{cases}
			x_1+x_2=-1\quad \Rightarrow |x_1+x_2|=1 \qquad &(i) \\
			x_1=-1-x_2\quad \Rightarrow |x_1|=|1-x_2| \qquad &(ii)\\
			x_2=-1-x_1\quad \Rightarrow |x_2|=|1+x_1| \qquad &(iii)
			\end{cases}
			$
			\end{center}
			So $u:= |x_1|, v:=|x_2|$ implies that 
			\begin{align*}
				1 &\leq u+v \\
				u &\leq 1+v \\
				v &\leq 1+u.
			\end{align*}
			So $(u,v) \in $
			\[ \text{ Show graph  of the rect in diag}\]
			then by taking the $\log$ of the figure, we would obtain the $Amoeba(1+x_1+x_2)=(\log|J_1|,\log|J_2|)$.
						\[ \text{ Show graph  of Amoeba}\]
			
			\underline{Q\&A:} What is the relation of the product $fg$ with $Newt(fg)$ and $ArchNewt(fg)$.
			\begin{align*}
			Newt(fg) &= Newt(f)+Newt(g) \\
			ArchNewt(fg) &= \ldots:::\text{mess}:::\ldots.
			\end{align*}
			It is a mess because $ArchNewt((1+x)^3) \neq 3ArchNewt(1+x).$
\end{document}
