\documentclass[]{article}
\usepackage[margin = 1.5in]{geometry}
\setlength{\parindent}{0in}
\usepackage{amsmath}
\usepackage{amsthm}
\usepackage{amsfonts}
\usepackage{amssymb}
\usepackage{mathtools}
\usepackage{wasysym}
\usepackage{soul}
\usepackage[T1]{fontenc}
\usepackage{ae,aecompl}
%\usepackage{tikz}
\usepackage[lined]{algorithm2e}
\usepackage{stackengine} %2 lines 1 text
\usepackage{enumerate}
%\usepackage{pstricks-add}
\usepackage{tikzsymbols} %For emoji's
\usepackage{pgf,tikz}\usepackage{mathrsfs}\usetikzlibrary{arrows}%\pagestyle{empty}
\usepackage[super,negative]{nth} %nth
\usepackage[pdftex,
  pdfauthor={Carlos Osco Huaricapcha},
  pdftitle={Solving System of Polynomial Equations},
  pdfsubject={Lecture notes from Algebraic Topology at MSRI},
  pdfkeywords={course notes, notes, Mathematics Science Research Institute, System of Polynomials Equations, combinatorics, mathematics},
  pdfproducer={LaTeX},
  pdfcreator={pdflatex}]{hyperref}

\usepackage{cleveref}

\newtheorem*{theorem}{Theorem}
\newtheorem*{corollary}{Corollary}
\newtheorem*{proposition}{Proposition}

\theoremstyle{definition}
\newtheorem{problem}{Problem}[section]
\newtheorem*{defn}{Definition}
\newtheorem{ex}{Example}[section]

\newcommand{\union}{\cup}
\newcommand{\intersection}{\cap}

\setlength{\marginparwidth}{1.5in}
\newcommand{\lecture}[1]{\marginpar{{\footnotesize $\leftarrow$ \underline{#1}}}}

\DeclarePairedDelimiter{\set}{\lbrace}{\rbrace}

\crefname{ex}{Example}{Examples}
\crefname{proposition}{Proposition}{Propositions}

\definecolor{darkish-blue}{RGB}{25,103,185}

\hypersetup{
    colorlinks,
    citecolor=darkish-blue,
    filecolor=darkish-blue,
    linkcolor=darkish-blue,
    urlcolor=darkish-blue
}


\tikzset{
	dot/.style 2 args={circle,inner sep=2pt,fill,label=#1:{#2},name=#2},
	rdot/.style 2 args={circle,inner sep=2pt,fill=red,label=#1:{#2},name=#2},
	gdot/.style 2 args={circle,inner sep=2pt,fill=green,label=#1:{#2},name=#2},
	bdot/.style 2 args={circle,inner sep=2pt,fill=blue,label=#1:{#2},name=#2}
}

\makeatletter
\def\blfootnote{\gdef\@thefnmark{}\@footnotetext}
\makeatother

\begin{document}
	\let\ref\Cref
	\title{\bf{Solving System of Polynomial Equations}}
	\date{Summer 2017, Mathematics Science Research Institute \\ \center Notes written from Maurice Rojas' lectures.}
	\author{Carlos Osco Huaricapcha}

	\blfootnote{I'd love to hear your feedback. Feel free to email me at \href{mailto:coscohua@mail.sfsu.edu}{coscohua@mail.sfsu.edu}.}

	\blfootnote{See \href{https://github.com/icarlitoss/MSRI-UP}{git:icarlitoss/msri-up} for updates.}


	\maketitle
	\newpage
	\tableofcontents
	\newpage

	\section{Introduction, Algebraic Geometry} \lecture{June 26, 2017}


		\subsection{What is it?}
		Ideally, Algebraic Geometry occurs over rings other than $\mathbb{C}$. Let's see some examples of polynomial solving.
			\begin{ex}
			Let $N=pq$ where $N$ is a positive integer and $p,q >1$. (Factoring.)
			Usually, we are given $N$, so we can find $p$ and $q$.\\
			The security of the R.S.A. Crypto-system is based on this \emph{hardness}.
			\end{ex}
			\begin{ex}
			Equilibria in Chemical Analysis systems (reaction networks).
			\begin{align*}
			\text{linear ordinary differential equation} \to &&\dot{x} = 3x+7\\
			\text{non-linear ordinary differential equation} \to &&\dot{x} = \text{polynomial}(x) (\text{deg}\geq 2)
			\end{align*}
			\end{ex}
			Differential equations govern chemical reaction rates. So, when the reaction settles down the concentrations reactions an \emph{equilibrium}, and the concentrations wind up being \emph{real} solutions to a system of polynomial equations.
			It is fair to say that \emph{real solving} in fact makes up to more than 30\% of electrical engineering.
			\begin{ex}
			$\mathbb{F}_1$: Finite field with $q$ elements in coding theory and cryptography. You often do arithmetic over $\mathbb{F}_q$, and polynomial system solving occurs very often. ($q$=prime power).
			\end{ex}
			\begin{ex}
			$\mathbb{Q}_p$($p$-acid reactions): This field is related to the solution of Weil-conjectures around the mid-20th century. Multiple fields medals were ensued.\\
			\end{ex}
			Fun fact: 40\% of the field medals are awarded to Algebraic Geometry.
			\\
			Let's start with easy equations (1 equation, 1 variable, 2 terms, $\mathbb{C}$).
			\begin{ex}
			Given $c_1,c_2 \in \mathbb{C}; a_1,a_2 \in \mathbb{Z}$. How do we solve
			\[ c_1x_1^{a_1}+c_2x_1^{a_2}=0\]?
			Notice that it is easy to reduce to the case
			\[ x^d=c \quad \text{where} \quad d\in\mathbb{Z}; c\in \mathbb{C}.\]
			In fact if you throw-out $x_1=0$, then we may assume $c\neq0$.
			\end{ex}
			%Eulers
			\subsection{Euler's Formula}
			It is easy to take d\textsuperscript{th} roots.
			\[ \exp^{\sqrt{-1}\tau}= \cos\tau+\sqrt{-1}\sin\tau.\]
			\textbf{Exercise 1.}
			What is a simple formula for the roots of $x_1^{d}=c$ using Euler's formula?
			\underline{Important observation:} Polynomials and polytopes are long lost siblings.
			%ArcNewt
			\subsection{ArchNewt and Newt}
			\begin{defn}
			$ArchNewt(c_1x_1^{a_1}+c_2x_1^{a_2}$):=
			%Image 1
			\definecolor{qqqqff}{rgb}{0,0,1}
			\begin{center}
			\begin{tikzpicture}[line cap=round,line join=round,>=triangle 45,x=1cm,y=1cm]\clip(-3.617,1.8798) rectangle (4.757,5.4202);\draw [line width=1.2pt] (-0.78,3)-- (2.42,4.36);\draw (-3.3213801765105226,2.673503804347824) node[anchor=north west] {$(a_1,-\log|c_1|)$};\draw (2.0793665987780043,5.328803804347821) node[anchor=north west] {$(a_2,-\log|c_2|)$};\begin{scriptsize}\draw [fill=qqqqff] (-0.78,3) circle (2.5pt);\draw [fill=qqqqff] (2.42,4.36) circle (2.5pt);\end{scriptsize}\end{tikzpicture}%%			\psset{xunit=1cm,yunit=1cm,algebraic=true,dimen=middle,dotstyle=o,dotsize=5pt 0,linewidth=0.8pt,arrowsize=3pt 2,arrowinset=0.25}\begin{pspicture*}(-4.3,-8.46)(22.34,6.3)\psline[linewidth=1.2pt](-0.78,3)(3.54,4.5)\rput[tl](-2.58,2.82){$(a_1,-\log|c_1|)$}\rput[tl](4.1,5.48){$(a_1,-\log|c_1|)$}\begin{scriptsize}\psdots[dotstyle=*,linecolor=blue](-0.78,3)\psdots[dotstyle=*,linecolor=blue](3.54,4.5)\end{scriptsize}\end{pspicture*}
			\end{center}
			%End Image
			Consider the slope of this line segment(assuming $a_1\neq a_2$).
			\begin{align*}
			\frac{\log|c_2|-(-\log|c_1|)}{a_2-a_1} = \log\left| \text{all the roots of } c_1x_1^{a_1}+c_2x_1^{a_2}\right|.
			\end{align*}
			\end{defn}
			%Exercise 2
			\textbf{Exercise 2.} Prove it.\\
			
			Here is another connection: If $f(x)=c_0+c_1x_1+\ldots+c_dx_1^d.$ where $d\geq1$ is the degree then $f$ has exactly $d$ roots counting multiplicity. (A version of the Fundamental Theorem of Algebra ``F.T.A'').
			This fact goes back to 1609 ``P. Roth,'' and 1629 ``Argand.''
			%%Newt
			\begin{defn}
			$Newt(c_o+c_1x_1+\ldots+c_dx_1^{d}):= [0,d].$
			%Pic 2
			\begin{center}
			\definecolor{qqqqff}{rgb}{0,0,1}\begin{tikzpicture}[line cap=round,line join=round,>=triangle 45,x=1cm,y=1cm]\clip(-8.837723278231774,1.6254902739899337) rectangle (16.07777765411344,7.062513608738572);\draw [line width=1.2pt] (-7.982588748474951,6.591950637700944)-- (-4.982588748474951,6.591950637700944);\draw (-8.123861965704762,7.013968757535459) node[anchor=north west] {$c_0$};\draw (-5.251005464071666,7.02090373627876) node[anchor=north west] {$c_dx_1^d$};\draw (-8.123861965704762,6.577065096707443) node[anchor=north west] {$0$};\draw (-5.1639492064464205,6.577065096707443) node[anchor=north west] {$d$};\begin{scriptsize}\draw [fill=qqqqff] (-7.982588748474951,6.591950637700944) circle (2.5pt);\draw [fill=qqqqff] (-4.982588748474951,6.591950637700944) circle (2.5pt);\end{scriptsize}\end{tikzpicture}
			\end{center}
		
			\underline{Observe:} $d=Length(Newt(\ldots))$.
			\end{defn}
			\subsection{Backup}
			\subsubsection{Convex}
			\begin{defn}
			Given any $S \subseteq \mathbb{R}^n$. We say $S$ is convex $\iff$ for all $x,y \in S$ the line segment connecting $x$ and $y$ also lies in $S$.
			\[ \{ \lambda x + (1-\lambda)y \mid \lambda \in [0,1]\}.\]
			\end{defn}
			\subsubsection{Convex Hull}
			\begin{defn}
			Given points $a1,\ldots,a_\tau \in \mathbb{R}^n$ their convex hull is
			\begin{align*}
			Conv(\{a1,\ldots,a_\tau\}):= \text{smallest convex set containing} a1,\ldots,a_\tau.
			\end{align*} 
			\end{defn}
			\begin{ex}
			$Conv(\{\ddots\}$) = Line.\\
			In $\mathbb{R}^3$ it is the wrapping ribbons around one point to another (?).
			\end{ex}
			\subsubsection{Polytope}
			\begin{defn}
			A polytope (in $\mathbb{R}^n$) is just the convex hull of any finite point set in $\mathbb{R}^n$.
			\end{defn}
			\subsubsection{Correct definitions for Newt and Archnewt}
			Let's think about column vectors. If $c_1x^{a_1}+\dots+c_Tx^{a_\tau} \quad (c_1,\dots,c_\tau \in \mathbb{C}^{*})$ where $X=(x_1,\dots,x_n)$ and $X^{a_{J}}=x_1^{a_1,J},\dots, X_n^{a_n,J}$. Then,
			\begin{align*}
			Newt(f) &: = \text{Conv}(\{a_1,\ldots,a_\tau\}) \\
			ArchNewt(f) &: = \text{Conv}(\{(a_J,-\log|c_J|) \mid \text{\scriptsize{J}} \in \{1,\ldots,\tau\}\}).
			\end{align*}
			\textbf{Exercise 3.} Find each of following: $Newt(0), Newt(1-x_1), Newt(1-x_1+x_2+x_1x_2)$, and $ArchNewt(1+1000x_1+x_1^3)$.
						\vspace{.25in}
			\\
			\underline{Note:} 
			\begin{align*}
			\mathcal{A} &:= [a_1,\ldots, a_\tau]\\
			&= \left[\begin{array}{ccc}a_{1,1} & \ldots & a_{1,\tau} \\\vdots & \ddots & \vdots \\a_{n,1} & \dots & a_{n,\tau}\end{array}\right] \in \mathbb{Z}^{n\times\tau}
			\end{align*}
			is called the support of $f$.
			\subsection{Univariate Trinomials}
			Let $f(x_1)=1-1000x_1^{18}+x_1{51}$. What can we say about the $(\mathbb{C})$ roots?
			
			\begin{center}``Physicists Approach\ldots'': $|x_1|
			\begin{cases}
			\text{small}\\
			\text{big}
			\end{cases}
			$\end{center}
			
			When $|x_1|$ is \emph{small:} $|x_1|^{18} \gg|x_1|^{51}$. Perhaps, the roots of $1-1000x_1^{18}$ are close to the roots of $1-1000x_1^{18}+x_1^{51} \Rightarrow$ maybe (?) there are \underline{18} roots of $f$ with norm $\sqrt[18]{\frac{1}{1000}}$ (small radius).
			
			Also, when $|x_1|$ is \emph{big:} $|x_1|^{51}\gg|x_1|^{18} \Rightarrow$ maybe(?) the nonzero roots of $1-1000x_1^{18}+x_1^{51}$ are near the nonzero roots of $f$. Perhaps, $f$ has 33 roots of norm near $\sqrt[33]{1000}$ (big radius).
			\\
			
			\underline{Truth:} 18 roots near inner orbit and 33 in the bigger orbit.\\
			
			\vspace{.25in}
				pic goes here of the orbits			
			\vspace{.25in}\\
			
			\textbf{Exercise 4.} How close are the norms of the roots of $f$ to the 2 approximations?\\
			
			\underline{Observe:} Archnewt$(f)$ is the following graph closed by the points $(0,0),(51,0),$ and $(18,-\log1000)$.\\
			
			\vspace{.25in}
			Graph of the triangle goes here
			\vspace{.25in}\\
			slope ``$s$'' of the lowets segements are close to the $\log|\text{roots of $f$|}$.
			
			\section{Archimedean Tropical Variety}
			\subsection{ArchTrop and Amoeba}
			The connection between polynomials and polytopes is closely related to Tropical Geometry. Historically, the roots of the connection go back to Newton around 1676, I. Newton wrote a letter describing how to extract power series solutions $y=y(x)$ to polynomial equations like $f(x,y)=0$.\\
			Also, J. Hadamard around 1896, derived a version of $ArcNewt$ for the power series, and observed a connection between slopes of edges and location of roots.
			\begin{defn} For one variable:
			\begin{align*}
			ArchTrop(f)= \text{slopes of lower edges of $ArchNewt(f)$}.
			\end{align*}
			\end{defn}
						\begin{defn}
			\begin{align*}
			Amoeba(f)= \{\log|J| \mid J \in \mathbb{C}, f(J)=0, J\neq 0\}.
			\end{align*}
			\end{defn}
			\begin{theorem}[Avenda\~{n}o, Kogan, Nisse, Rojas.(2013)] If $f(x_1)=c_1x^{a_1}+\dots+c_\tau x^{a_\tau}$ then, for any $u \in Amoeba(f)$ there is a $v\in ArchTrop(f)$ with $|u-v| \leq \log 3$.
			\end{theorem}
			 I.e. These sets approximate each other.
			\begin{ex} Given: $T=2$(binomials). Then, $Amoeba = $ 1 point and $ArchTrop=$ 1 point.
			\end{ex}
			\begin{ex} Given: $T=3$(trinomials). Then, $Amoeba = $ up to the degree many points points. $1-1000x^{18}+x^{51}$ has $\approx$ 26 points (?) . Also, $ArchTrop(f)=$ 2 points. (Include graph of numb line).
			\end{ex}
			\textbf{Exercise 5.} Figure out how many points.\\
			\subsection{Vertex, Edge, and Lower Edge}
			\subsubsection{Vertex}
			\begin{defn}
			Consider minimizing $\alpha x+\beta y$ for $(x,y)\in P.$ If this minimum is attained at a unique point $v$ then we call $v$ a \emph{vertex} of $P$ with inner normal $(\alpha,\beta)$.
			\end{defn}
			\subsubsection{Edge}
			\begin{defn}
			Consider minimizing $\alpha x+\beta y$ for $(x,y)\in P.$ If this minimum is attained at a proper subset of $P$(with $\geq$2 points) then this set is called an \emph{edge} with inner normal $(\alpha,\beta)$. (Include graph of norms).
			\end{defn}
			\subsubsection{Lower Edge}
				\begin{defn}
			If $P \in \mathbb{R}^2$ is a polygon then $E \subset P$ is a lower edge if and only if $E$ is an edge with inner normal $(\alpha,\beta)$ for some $(\beta >0)$. (Include graph of polygon).
			\end{defn}
			\section{ArchTrop and Amoeba in multivariable functions}
			Consider $f(x_1,x_2)=1+x_1+x_2$.
			\begin{defn}
			If $f$ is any polynomial in $x=(x_1,\ldots,x_n)$ then 
			\[Amoeba(f):= \{(\log|J|,\ldots,\log|J_n|) \mid f(J_1,\dots J_n)=0; J_1,\dots J_n \in \mathbb{C}^{*} \}.\]
			\end{defn}
			\begin{defn}
			%\begin{align*}
			\[ArchTrop(f):= \{v\in \mathbb{R}^{n} \mid (v_1,-1) \text{ is an outer normal to a face of $ArchNewt(f)$ of positive dimension}\}.\]
			%\end{align*}
			\end{defn}
			\underline{Note:} For $n=1$, ($ArchNewt \subset \mathbb{R}^{2}$) a lower edge has slope $v \iff$it has an outer normal of the form $			(v,-1)$.
			\[ \text{put picture of the convace outter normal}\]
			\underline{Note:} We will clarify faces and dimension in complete generality next time. For now, let`s consider the $Amoeba$ .
			\[ \text{Enter pic of Amoeba here}\]
			Amoebae are useful because they give some intuition to begin higher dimensional zero sets.\\
			\underline{Note:} $f(x_1,x_2)$ where $x_1,x_2 \in \mathbb{C}$ has 2 manifold in $\mathbb{R}^{+}$.\\
			Let's analyze why is that picture true.
			\begin{center}
			If $ 1+x_1+x_2=0$ then $\begin{cases}
			x_1+x_2=-1\quad \Rightarrow |x_1+x_2|=1 \qquad &(i) \\
			x_1=-1-x_2\quad \Rightarrow |x_1|=|1-x_2| \qquad &(ii)\\
			x_2=-1-x_1\quad \Rightarrow |x_2|=|1+x_1| \qquad &(iii)
			\end{cases}
			$
			\end{center}
			So $u:= |x_1|, v:=|x_2|$ implies that 
			\begin{align*}
				1 &\leq u+v \\
				u &\leq 1+v \\
				v &\leq 1+u.
			\end{align*}
			So $(u,v) \in $
			\[ \text{ Show graph  of the rect in diag}\]
			then by taking the $\log$ of the figure, we would obtain the $Amoeba(1+x_1+x_2)=(\log|J_1|,\log|J_2|)$.
						\[ \text{ Show graph  of Amoeba}\]
			
			\underline{Q\&A:} What is the relation of the product $fg$ with $Newt(fg)$ and $ArchNewt(fg)$.
			\begin{align*}
			Newt(fg) &= Newt(f)+Newt(g) \\
			ArchNewt(fg) &= \ldots:::\text{mess}:::\ldots.
			\end{align*}
			It is a mess because $ArchNewt((1+x)^3) \neq 3ArchNewt(1+x).$
			%%%Day2
			\subsection{Orientation} \lecture{June 27, 2017}
			\begin{defn}An \emph{orientation} of a line $L (\subset \mathbb{R}^{2})$ is just attaching a nonzero vector $v(\in\mathbb{R}^2)$ to $L$.
			\end{defn}
			\[
			\text{ line x+y-5}
			\]
			\underline{Note:} This allow us to speak of \emph{left} and \emph{right}.
			\subsection{Puzzle}
			\begin{defn}
			Given an oriented line $L$(through $\overrightarrow{o}$) and a point $P_1$.
			\end{defn}
			How do we determine if $P$ belongs to left of $L$ or the right $L$?
			$L=x-$axis, orientation vector $v=(1,0).$
			\[\text{Ray oriented pic goes here}\]
			\begin{ex} What side is $P$ on? (Generalize).\\
			Sign of the determinant would determine if it is right or left.
			\begin{equation*}
			\text{if }\det[v,p]\begin{cases}
			>0 \quad \iff p\text{ is left of } L.\\
			=0 \quad \iff p\in L.\\
			<0 \quad \iff p\text{ is right of } L.
			\end{cases}
			\end{equation*}
			\end{ex}
			%%%%% Incremental Algorithm
			\section{An Incremental Algorithm}
			To complete $Conv$ in $\mathbb{R}^2$ is a basic problem in computational Geometry. Here is a simple way to do it.\\
			Input: $P_1,\ldots,P_N \in \mathbb{R}^2$.
			Output: A list of indices $\{c_1,\ldots,c_k\}$.
			such that:\\
			$(0):$ the vertices $Conv\{P_1,\ldots,P_N\}$ are exactly $P_i,\ldots,P_{i_k}.$\\
			$(1):$ $P_i,\ldots,P_k$ are ordered $C.C.W.$\\
			
			\subsection{Description}
			\begin{itemize}
			\item $(-1):$ If $P_1=\ldots=P_N$ then $Conv=P_1$ and you are done \Smiley.\\
			
			\item $(0):$ If $\det[P_2-P_1,P_i-P_1]=0$ (Assume $P_2-P_1\neq 0$) for all $i\geq 3$ then let $P_{i_1}$ be any point.
				\begin{itemize}
			\item minimizing: $P_{i_1}\cdot(P_2-P_1)$, then $P_{i_2}$ be any point.
			\item maximizing: $P_{i_2}\cdot(P_2-P_1)$, then $Conv =$ the line segment connecting $P_{i_1}$ and $P_{i_2}$.
						\end{itemize}
			\item $(1):$ Find 3 points $\overbrace{P_1,P_2,P_3}^{W\log}$ with $\det(P_2-P_1,P_3-P_1) \neq 0.$
			i.e. Forming a non-degen triangle.
			\item $(2):$ Let $S=\{1,2,3\}$. Let $J:= 4$
				\begin{enumerate}[(a)]
				\item Take $P_J$ and check if $P_J \in Conv$ defined by $S$.
				\item If not find the uniques edges $\overline{P_J-P}$ and $\overline{P_JP_v}$ of $Conv((\bigcup_{l\in S} P_{l})\cup P_J)$
				\end{enumerate}
			\item $(3):$ Update $S, J, J=J+1.$ Go to $(2)$ unless no points left.
			\item $(4):$ Output $S$.
			\end{itemize}
			\begin{ex} Let`s go back to the beginning, and trace the convex hull.
			\end{ex}
						\[ \text{Graph of the progress goes here}\]
			%%%%Complexity
			\subsection{Complexity}
			I.e. how much work? and how do you measure?\\
			Input size: N(\# of points).\\ Work: \# of Field operations $(+,-,\times,\div)$ and sign checks involving $\mathbb{R}$ numbers.\\
			Measuring this way, we can see that steps in $(-1)$ and $(0)$ use:
			\begin{enumerate}[(i)]
				\item $\leq 2(N-1)$ = checks.
				\item $\leq 2$ subtractions of vectors.
				\item $N-1$ determinants and sign checks.
			\end{enumerate}
			Hence, it takes about $2+3(N=1)$ operations in the end. $\Rightarrow O(\mathcal{N})$ops.\\
			The remaining steps then involve:
			\begin{enumerate}[(i)]
				\item 3 left and right checks, update list.
				\item $<4$ left and right checks.
				$\vdots$
				\item $N-1$ left and right checks.
			\end{enumerate}
			Hence,$\Rightarrow O(\mathcal{N}^2)$ operations in the end. $\Rightarrow O(\mathcal{N}^2)$ops . Therefore, it implies that $O(\mathcal{N})$ complexity required.\\
			This algorithm is from the 1980`s and, while not the fastest, it is conceptually one of the easiest. Also easy to extend to arbitrary dimensions.\\
			
			\underline{Note:} There are $O(\mathcal{N}\log\mathcal{N})$ convex hull algors (in $\mathbb{R}^2$). It turns out that every algorithm (for $Conv \in \mathbb{R}^2$) has worst-case complexity $\Omega(\mathcal{N}\log\mathcal{N})$.\\
			In dimension $d$, Chazelle and Edelsbrunner proved in the 1980`s that $Conv(N\text{-points})$ admits a $O(N^{\lfloor d/2 \rfloor}+N\log N)$ algorithm, and this is optimal.\\
			\subsection{Alternative Definition for ArchTrop(f)}
			\begin{defn}
			If $f(x)=c_1x^{a_1} +\ldots+c_\tau x^{a_\tau}$.
			\begin{align*}
			ArchTrop(f): \left\{ v \in \mathbb{R}^n : \stackanchor{max}{\tiny{$J\in\{1,\ldots,\tau\}$}} \mid c_{J}e^{a_J\cdot v} \mid \text{ is attained at }\geq 2 \text{ distinct points}\right\}.
			\end{align*}
			\end{defn}
			\underline{Note:} Tropical Geometry is nothing more than checking when 2 things agree.
			\\
			\textbf{Exercise 1.} Binomial case in $n$ variables. Check that definition via $ArchNewt$ = definition via max.
			\begin{ex}
			Given: $1-1000x_1^{18}+x_1^{51}.$ There are 3 terms. Then 3 possibilities.
			\begin{enumerate}[(i)]
			\item $|1|=|-1000e^{18v}| \quad > |e^{51v}|$.
			\item $|-1000e^{18v}| = | e^{51v}| \quad > |1|$
			\item $|e^{51v}| = 1 \quad |-1000e^{18v}$.
			\end{enumerate}
			Then, solving each case we would get that 
			\begin{enumerate}[(i)]
			\item $v= \frac{1}{18} \log 1000$ (and $e^{51v}$ is simply smaller).
			\item $v = \frac{1}{33} \log 1000$ (and$ e^{18v}>1$.
			\item $v=0$ (but $1>1000$), therefore we exclude it.
			\end{enumerate}
			\end{ex}
			\subsection{Back up}
			\subsubsection{Affine}
			\begin{defn}
			An \emph{affine} subspace $S$ of $\mathbb{R}^n$ is just a translate of a (linear) subspace of $\mathbb{R}^n$.
			\end{defn}
			\subsubsection{Face}
			\begin{defn}
			A \emph{face} Q of a polyhedron $P \subseteq \mathbb{R}^n$ is just equivalently:
			\begin{enumerate}[(a)]
			\item $P \bigcap $ supporting hyperplane.
			\item $\{x \in P | v\cdot x \text{ is minima}l\}$ for some inner normal $v$. 
			\item $\{x \in P | w\cdot x \text{ is maximal}l\}$ for some outerr normal $w$.
			\end{enumerate}
			\end{defn}
			\subsubsection{Half Space}
			\begin{defn}
			A \emph{half space} $\subset\mathbb{R}^n$ is just
			$H=\{x | x\cdot x \leq c\}$
			for some $v$,
			\end{defn}
			\subsubsection{Dimension}
			\begin{defn}
			The \emph{dimension} of $Q$ is just the dimension of the smallest affine subspace.
			\end{defn}

			

%
%
%
%
%
%
%
%
%
%
%
%
%
%
%			
\end{document}